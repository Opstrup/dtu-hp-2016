\documentclass[11pt]{article}
\usepackage{fullpage}
\usepackage{amsmath, amsfonts}
\usepackage[utf8]{inputenc}

\begin{document}
\begin{center}
{{\Large \sc Computationally Hard Problems}}
\end{center}
\rule{\textwidth}{1pt}
\begin{description}
\item[Student name and id:] Anders H. Opstrup (s160148)
\item[Collaborator name(s) and id(s):]
\item[Hand-in for week:] 3
\end{description}
\rule{\textwidth}{1pt}

\section*{Exercise 1}
A yes-no-problem is in $RP_{1/10}$ if there is a polynomial $p$ and a randomized $p$-bounded algorithm A such that for every input X the following holds: \newline

True answer for X is YES then $P_{R}$[A(X, R) = YES] $>=$ 1/10

True answer for X is NO then  $P_{R}$[A(X, R) = NO] $>=$ 1

\subsection*{a)}
Prove that $RP_{1/10} = RP$ and $RP\subseteq BPP$

$RP_{1/10} = RP$

$RP\subseteq BPP$ \newline

If the definition of class $A$ is a restriction of the definition of class $B$ then we have $A \subset B$. \newline

RP-algorithms (Monte Carlo-algorithms) have one-sided error. Which means they have a pretty good change of getting the correct result, namely at least 50\%. 
Taken that in consideration we can there by conclude that when running the $RP_{1/10}$ 5 times we will get a success rate of 50\%, there by showing that $RP_{1/10} = RP$. \newline

The BPP algorithm's probability of answering correct should be strictly greater than 1/2. The success rate is determined by the constant $\epsilon$. To proof that $RP\subseteq BPP$ we can boot the success probability of the RP-algorithm to at least $1/2 + \epsilon$ to show that $RP\subseteq BPP$.

\end{document}