\documentclass[11pt]{article}
\usepackage{fullpage}
\usepackage{amsmath, amsfonts}
\usepackage[utf8]{inputenc}

\begin{document}
\begin{center}
{{\Large \sc Computationally Hard Problems}}
\end{center}
\rule{\textwidth}{1pt}
\begin{description}
\item[Student name and id:] Anders H. Opstrup (s160148)
\item[Collaborator name(s) and id(s):]
\item[Hand-in for week:] 7
\end{description}
\rule{\textwidth}{1pt}


\section*{Exercise 1}

Show the computation of $[\frac{1543}{799}]$ using the rules shown in the lecture notes. You may use that gcd(1543, 799) = 1.

\begin{enumerate}
	\item $[\frac{1543}{799}]$ = $[\frac{744}{799}]$				by I-2
	\item = $[\frac{2}{799}]$ , $[\frac{372}{799}]$ 					by I-1
	\item = $[\frac{2}{799}]$ , $[\frac{186}{799}]$ 					by I-1
	\item = $(+1) [\frac{186}{799}]$ 											by I-5 ($799 \equiv 7 mod 8$)
	\item = $(+1) [\frac{55}{799}]$												by I-2
	\item = $(+1)(-1) [\frac{799}{55}]$											by I-3
	\item = $[\frac{29}{55}]$															by I-2
	\item = $(-1) [\frac{55}{29}]$													by I-3
	\item = $(-1) [\frac{26}{29}]$													by I-2
	\item = $(-1)[\frac{2}{29}]$, $[\frac{13}{29}]$						by I-1
	\item = $(-1)(-1)[\frac{29}{13}]$												by I-3
	\item = $(-1)(-1)[\frac{3}{13}]$												by I-2
	\item = $(-1)(-1)(-1)[\frac{13}{3}]$											by I-3
	\item = $(-1)(-1)(-1)[\frac{1}{3}]$											by I-2
	\item = $(-1)^{2}(-1)(+1)$
	\item = $(-1)^{2}$
	\item = $1$
\end{enumerate}

\end{document}
