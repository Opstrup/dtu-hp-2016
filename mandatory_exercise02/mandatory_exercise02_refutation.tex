\documentclass[11pt]{article}
\usepackage{fullpage}
\usepackage{amsmath, amsfonts}
\usepackage[utf8]{inputenc}

\begin{document}
\begin{center}
{{\Large \sc Computationally Hard Problems}}
\end{center}
\rule{\textwidth}{1pt}
\begin{description}
\item[Student name and id:] Anders H. Opstrup (s160148)
\item[Collaborator name(s) and id(s):] Salik Lennert Pedersen (s134416)
\item[Hand-in for week:] 2
\end{description}
\rule{\textwidth}{1pt}


\section*{Exercise 1}
Given is a disjunctive form consisting of $k$ monomials $m_{1}, . . . , m_{k}$ over $n$ boolean variables $x_{1}, . . . , x_{n}$. The task is to decide if there is a truth assignment to the variables such that the truth value of the disjunctive form is false.
\newline \newline
You are given a decision algorithm $A_{d}$ that solves this problem, i. e., for each instance to REFUTATION, $A_{d}$ answers YES if there is an assignment that makes the truth value of the disjunctive form false; otherwise it answers NO.

\subsection*{a)}
Describe an algorithm $A_{o}$ which solves the optimization problem, that is, which finds a truth assignment making the disjunctive form false if one exists. The running time has to be polynomial in the input size and $A_{o}$ may make calls to $A_{d}$. Such calls count as one basic computational step.
 
\subsection*{b)}
Argue that your algorithm is correct.

\subsection*{c)}
Prove that the running time of the algorithm is bounded from above by a polynomial.Any polynomial is sufficient; you need not look for a polynomial of minimal degree. Recall that a call to $A_{d}$ counts one step.
\newline \newline
\textbf{Note:} The input of REFUTATION is disjunctive form of monomials over $n$ boolean variables, nothing else. In particular, a legal input cannot specify specific settings of variables.

\end{document}